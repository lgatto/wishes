% \documentclass[preview]{standalone}
\documentclass{article}
\usepackage{varwidth}
\usepackage[fleqn]{amsmath}
\setlength\mathindent{0pt}
\usepackage{url}
\usepackage{textcomp}

\title{A Formal Definition of Season's Greetings}

\author{Lisa M. Breckels$^*$
  and Laurent Gatto\footnote{Computational Proteomics Unit, University
    of Cambridge Tennis Court Road Cambridge, CB2 1GA, UK}}


\begin{document}

\maketitle

\section*{Introduction}

In this work, we provide a formal demonstration of the yearly season's
greetings. In addition, we provide an interactive 3 dimensional
visualisation of a \textit{hyperLOPIT} experiment\cite{LOPIT}.

\section*{Material and methods}

The 3D specimen was printed on a MakerBot Replicator 2. Hardware
access was kindly provided by Dr. B Adryan. Colour of the specimen was
performed using \textit{essentials}\texttrademark Acrylic Artist
Colours generously provided by Dr. M Deery.

\section*{Demonstrations}

\subsubsection*{Greetings 1}

\begin{varwidth}{\linewidth}
\begin{align*}
y &= \frac{\log_e \left(\frac{x}{m} - sa\right)}{r^2}\\
%
yr^2 &= \log_e \left(\frac{x}{m} - sa\right)\\
%
e^{yr^2} &= \frac{x}{m} - sa\\
%
me^{yr^2} &= x - msa\\
%
me^{rry} &= x - mas
\end{align*}
\end{varwidth}

\newpage

\subsubsection*{Greetings 2}

\begin{varwidth}{\linewidth}
\begin{align*}
\ln \left( \frac{e^{a_r} + p^2 H_a}{N} \right) &= w - \ln (y) \\
%
\ln (y) + \ln \left( \frac{e^{a_r} + p^2 H_a}{N} \right) &= w \\ 
%
\ln \left( \frac{e^{a_r} + p^2 H_a}{N} y \right) &= w\\
%
\frac{e^{a_r} + p^2 H_a}{N} y &= e^w\\
\left( H_a p^2  + e^{a_r} \right) y &= N e^w\\
H_a ppy &= N e^w - ye^{a_r}
\end{align*}
\end{varwidth}


\section*{Conclusions}

We anticipate further progress and groundbreaking results in 2016.

\bigskip

\subsubsection*{Acknowledgement}

We would like to thank Mr. R. Gatto for the cover art.

\begin{thebibliography}{1}

\bibitem{LOPIT} Christoforou A, Mulvey CM, Breckels LM, Geladaki A,
  Hurrell T, Hayward P, Naake T, Gatto L, Viner R, Martinez Arias A,
  and Lilley KS. {\em A draft map of the mouse pluripotent stem cell
    spatial proteome, Nature Communications}, Nature Communications,
  2015.

\end{thebibliography}

\end{document}